\chapter{The Neural Network used for signal-background classification}
\section{Short introduction to neural networks}
Neural networks are loosely on the human brain and are a means of doing machine learning, in which a computer learns to perform some task by analyzing training examples. 
A NN is usually organized into layers of processing nodes. These processing nodes are densely interconnected between layers, and every connection weighted. 
During the $forward$ $propagation$, where the NN is tested on given data, processes in the NN propagate from the input layer to the output layer. An individual node receives data from nodes in the layer beneath it and sends data to nodes in the layer above. 
Nodes multiply received data by their weight value and add them together to a single value. Only if the node exceeds a specific value does it send its value to the next layer. During $backpropagation$, where the NN is being trained, weights and thresholds are continually adjusted until training data with the same label yield similar results.   
In this thesis, a NN discriminates between the $tq\gamma$ signal and background events. 
The NN is trained on Monte Carlo simulations (Section \ref{sec:mc}) and is tested on measurement data after that. 
\section{The neural network architecture}
\label{sec:arch}
\begin{figure}
    \centering
    \begin{subfigure}{.5\textwidth}
      \centering
      \includegraphics[width=.4\linewidth]{Plots/model_0fj.png}
    \end{subfigure}%
    \begin{subfigure}{.5\textwidth}
      \centering
      \includegraphics[width=.4\linewidth]{Plots/model_1fj.png}
    \end{subfigure}
    \caption{Plots visualizing the NN architecture for the zero forward jet region (left) and the $\geq 1$ forward jets region (right).}
    \label{fig:models}
\end{figure}

\section{Input features for the neural network}
\label{sec:inputfeatures}
\begin{table}
    \centering
    \begin{tabular}{c|c|c}
        \toprule
        {} &                     0fj variables      & 1fj variables\\
        \midrule 
        1  &                                HT      & HT\\ \hline
        2  &                           blep\_dr     & blep\_dr\\ \hline
        3  &                           lbj\_eta     &lbj\_eta\\ \hline
        4  &                            lbj\_pt     &lbj\_pt\\ \hline
        5  &  lbj\_tagWeightBin\_DL1r\_Continuous   & lbj\_tagWeightBin\_DL1r\_Continuous\\ \hline
        6  &                          lep1\_eta     & lep1\_eta\\ \hline
        7  &                           met\_met     &met\_met\\ \hline
        8 &                             ph\_pt     &ph\_pt\\ \hline
        9 &                             top\_m     & top\_m\\ \hline
        10 &                       transMassWb      &transMassWb\\ \hline
        11  &                            bph\_pt     & bph\_m\\ \hline
        12 &                          topph\_pt     &topph\_ctheta\\ \hline
        13 &                            ph\_eta     & ph\_phi\\ \hline
        14 &                          lepph\_dr     & lep1\_pt\\ \hline
        15 &                      transMassWph      & met\_phi\\ \hline
        16 &                           lep1\_id     & lbj\_phi\\ \hline
        17 &&                                           Wbsn\_e \\ \hline
        18 &&                                            bfj\_m \\ \hline
        19 &&                                           blep\_m \\ \hline
        20 &&                                           fj\_eta \\ \hline
        21 &&                                           fj\_phi \\ \hline
        22 &&                                        fjet\_flag \\ \hline
        23 &&                                      fjph\_ctheta \\ \hline
        24 &&                                        fjph\_deta \\ \hline
        25 &&                                          fjph\_dr \\ \hline
        26 &&                                           fjph\_e \\ \hline
        27 &&                                           fjph\_m \\ \hline
        \bottomrule 
    \end{tabular}
    \caption{Input variables of the NN trained on events with no forward jets and the NN trained on events with at least one forward jet.}
    \label{tab:features}
\end{table}


\section{Performance and distribution of the NN output}

