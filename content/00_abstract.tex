\thispagestyle{plain}
\nocite{PhysRevLett.121.221802}

\section*{Abstract}
\begin{english}
The investigation of the top quark coupling to the photon offers a critical test of the Standard Model.
Differential measurement of the $tq\gamma$ process of the Standard Model would grant new insights into this electromagnetic coupling of the top quark. 
Regarding a measurement of $tq\gamma$ with the full Run-2 dataset at the ATLAS experiment, a neural network is trained on data from Monte Carlo simulations to discriminate background processes from the $tq\gamma$ signal process. 
Certain features of events are provided as input to the neural network. 
In this thesis, the influence of these input features on the neural network's ability to separate signals from the background is investigated, 
by first calculating correlations of the features with the NN output and then examining how different requirements on the features influence the composition of the output. 
This provides knowledge into which event features are essential characteristics to understand the electromagnetic coupling of the top quark.
\end{english}
