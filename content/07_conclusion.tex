\chapter{Conclusion}
The study in this thesis has been conducted regarding a differential measurement of the single top quark production in association with a photon of the Standard Model to closer examine the electromagnetic coupling of the top quark. 
Analyses of the electromagnetic coupling of the top quark provide important tests to the Standard Model. This anaylsis aimed to use a neural network, used for signal-background discrimination, to find variables sensitive to the top quark's EM coupling. 

The influence of different input variables on the NN has been thoroughly analyzed. It has been calculated how the characteristic features of the $tq\gamma$ process 
correlate with the output. Highly correlated and less correlated features are identified, and the reasons for high or low correlation are discussed. 
Most correlations follow the expectations, verifying the correctness of the calculations. 

Two input features, the transverse momentum of the photon $p_T^\gamma$ and the sum of the forward jet energies and the photon energy $E_{fj\text{+}\gamma}$, have been further examined. 
These features are examples of a strongly correlated and a weakly correlated variable to test the nature of the NN at borderline cases. This is useful to determine whether the NN needs adjustment. 
For $p_T^\gamma$, it has been found that higher divisions in energy, such as the division into the region $p_T^\gamma > 40 \,\si{\giga\electronvolt}$, result in a greater $S/B$ ratio in the NN output than lower energy divisions. 
While this observation is significant, it is not as substantial as the effects of divisions for the $E_{fj\text{+}\gamma}$ input feature. The much higher correlation of $E_{fj\text{+}\gamma}$ leads to more significant changes in the NN output for different divisions. 
For this feauture, it is again found that higher energies, such as the region $E_{fj\text{+}\gamma} > 900 \,\si{\giga\electronvolt}$, ensues a significantly larger $S/B$ in the NN output. 

This thesis provides an essential framework for the differential measurement of $tq\gamma$ with the full Run-2 dataset of the LHC. It suggests characteristic features that could be sensitive to the EM coupling of the top quark. 
In addition, it also confirms that the significance of divisions on input features coincide with the calculated correlations. 
The framework provided by this thesis may be used to inspect the input variables further. 
Regarding future studies, conducting the so-called $distance$ $correlation$ \cite{distcorr} may provide further indications to whether the NN needs to be adjusted. 