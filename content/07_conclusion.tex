\chapter{Conclusion}
The influence of different input variables on the NN output has been thoroughly analyzed. It has been calculated how the characteristic features of the $tq\gamma$ process 
correlate with the output. Highly correlated and less correlated features are identified, and the reasons for high or low correlation are discussed. 
Most correlations follow the expectations, verifying the correctness of the calculations.

Two input features, the transverse momentum of the photon $p_T^\gamma$ and the sum of the forward jet energies and the photon energy $E_{fj\text{+}\gamma}$, have been further examined. 
For $p_T^\gamma$, it has been found that higher divisions in energy, such as the division into the region $p_T^\gamma > 40 \,\si{\giga\electronvolt}$, result in more efficient signal-background separations (higher values for $\frac{S}{\sqrt{B}}$) in the NN output than lower energy divisions. 
While this observation is significant, it is not as substantial as the effects of divisions for the $E_{fj\text{+}\gamma}$ input feature. The much higher correlation of $E_{fj\text{+}\gamma}$ leads to more significant changes in the NN output for different divisions. 
For this feauture, it is again found that higher energies, such as the region $E_{fj\text{+}\gamma} > 1 \,\si{\tera\electronvolt}$, ensues substantially better seperation in the NN output. 

All in all, the goal of this thesis has been achieved successfully. This thesis provides an essential framework for the differential measurement of $tq\gamma$ with the full Run-2 dataset of the LHC. It provides an overview of different input variables concerning the influence on the NN output. 
Furthermore, it is shown that divisions on different input features are well suited for optimising of signal-background discrimination in the NN output. Additionally, it is also confirmed that the significance of divisions on input features coincide with the calculated correlations. 
The framework provided by this thesis may be used to inspect different input variables further. Due to the scope of the thesis, additional inspections of the input features with regard to the NN were not made.