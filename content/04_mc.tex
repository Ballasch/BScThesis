\chapter{Monte Carlo samples and event selection}


\section{Generation of Monte Carlo samples}
\label{sec:mc}

The simulation of a process is done in three steps: First, the event is generated by calculating corresponding matrix elements. Then the resulting electromagnetic and hadronic showers need to be modelled. Lastly, the detector itself needs to be simulated in order to account for the detector response and the reconstruction of events.
% \begin{enumerate}
%     \item First, the event is generated by calculating corresponding matrix elements.
%     \item Then the resulting showers need to be modelled.
%     \item Lastly, the detector itself needs to be simulated in order to account for the detector response and the reconstruction of events.
% \end{enumerate}

The framework \textsc{MadGraph5\_aMC@NLO} is used for Monte Carlo (MC) simulations of the considered $tq\gamma$ process. \textsc{MadGraph5} is a matrix element generator that allows the interfacing of different packages for further simulation. 
The simulated events are generated at next-to-leading order (NLO) at the $t$-channel of single top production. The generator is interfaced to the package \textsc{Pythia} $v8.240$, which provides parton showers. 
The \textsc{MadSpin} and \textsc{EvtGEN} $v1.6.0$ packages give decay simulations of the top and bottom quark, respectively. Here, only leptonic decays of the top quark are considered.

Moving on to background processes, the $t\bar{t}$ process is modelled at leading order (LO) also using \textsc{MadGraph5\_aMC@NLO} $v2.3.3$ interfaced to \textsc{Pythia} $v8.212$. 
Simulation of $W\gamma$+jets and $Z\gamma$+jets events are produced at NLO using the \textsc{Shepra} $v2.2.2$ and \textsc{Shepra} $v2.2.4$ packages. For the $t\bar{t}$ process and $t$-, $s$-, $tW$-channels 
\textsc{Powheg-Box} is used where \textsc{Pythia} $v8.230$ is again used as the showering program. The modeling here is performed in NLO in QCD. 

The final events generated by \textsc{Pythia} are processed through an ATLAS detector simulation build with the \textsc{Geant4} detector simulation toolkit \cite{Geant4}. 
This model of the ATLAS is reconstructs leptons, photons and jets from the detector response. To achieve this, the structure of the ATLAS detector as described in Section \ref{sec:atlas} is implemented the \textsc{Geant4} based simulation. This includes a simulation of the inner detector, the electromagnetic calorimeter, the hadronic calorimeter and the muon spectrometer. 
Events are then reconstructed by analysing particles passing through each detector component as described in \autoref{sec:reconstruction}. 

The table \ref{tab:eventgen} gives a summary of the generated samples and their generators.

\begin{table}
    \centering
    \begin{tabular}{c|c}
        \toprule
        Process & Generator\\
        \midrule
        $tq\gamma$&$MadGraph5\_aMC@NLO$ + $Pythia8$\\[.1cm]
        $t\bar{t}\gamma$&$MadGraph5$ + $Pythia8$\\[.1cm]
        $W\gamma + jets$&$Sherpa$ $2.2.2$\\[.1cm]
        $Z\gamma + jets$ &$Sherpa$ $2.2.4$\\[.1cm]
        $t\bar{t}$ &$Powheg$ + $Pythia8$\\[.1cm]
        single top&$Powheg$ + $Pythia8$\\[.1cm]
        $W+jets$& $Sherpa$ $2.2.1$\\[.1cm]
        $Z+jets$ &$Sherpa$ $2.2.1$\\[.1cm]
        Diboson &$Sherpa$ $2.2.2$\\
        \bottomrule
    \end{tabular}
    \caption{List of generated samples alongside their generators.}
    \label{tab:eventgen}
\end{table}
\section{Event selection}
\label{sec:eventselect}
The selection criteria for events must hold the necessary conditions for a $tq\gamma$-process. It also needs to have enough restrictions to reduce background contributions as much as possible. 
Signal events have precisely one lepton, at least one photon and one $b$-tagged jet in the final state. The lepton should have a transverse momentum higher than $20 \,\si{\giga\electronvolt}$, the photons momentum higher than $27\,\si{\giga\electronvolt}$ and 
the $b$-tagged jet has to pass the $DL1r$-algorithm with a $70\%$ working point.\\
Additionally, the missing transverse energy $E_T^{miss}$ ought to be above $30 \,\si{\giga\electronvolt}$ to account for the neutrino in the decay mode. 
Finally, to reduce leading background contributions from the $Z \rightarrow ee(\rightarrow \gamma)$ process, the invariant mass of the leading photon and an electron candidate $m_{e\gamma}$ is set to be in the range $80 \,\si{\giga\electronvolt} < m_{e\gamma} < 110 \,\si{\giga\electronvolt}$.
Altogether, this makes up the following requirements for selected events:
\begin{enumerate}
    \item At least one photon $\gamma$ with $p_T > 20 \,\si{\giga\electronvolt}$
    \item Exactly one lepton with $p_T >27\,\si{\giga\electronvolt}$
    \item $E_T^{miss} > 30 \,\si{\giga\electronvolt}$
    \item Exactly one $b$-tagged jet passing $70\%$ working point (WP) of the $DL1r$-algorithm. 
    \item Invariant mass of leading photon and electron candidate between values $80 \,\si{\giga\electronvolt} < m_{e\gamma} < 110 \,\si{\giga\electronvolt}$  
\end{enumerate}