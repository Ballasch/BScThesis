\chapter{Introduction}

The Standard Model (SM) of particle physics  describes the nature of discovered elementary particles and three of the four fundamental interactions: the strong, weak and electromagnetic interaction. Only the gravitational interaction is not covered by the theory. The SM has been researched extensively and is widely regarded as the most successful theory. 
However, the SM contains several conceptional problems and is unable to explain all phenomena. For instance, the existence of dark matter is not accounted for in the SM. The many problems of the SM motivate the search for physics beyond the Standard Model (BSM). 
It is therefore essential to test and research the limits of this theory. \\
Many tests of the Standard Model involve the top quark, the most massive elementary particle. One such test would be the search for the single production of a top quark in association with a photon in proton-proton collisions ($pp \rightarrow tq\gamma$). 
The process is sensitive to the electroweak coupling of the top quark. As this coupling is a critical parameter of the SM, a precise measurement of the cross-section for this process may give new insights into this parameter. 
The $tq\gamma$ process has not yet been observed but the CMS Collaboration reported evidence for it in 2018 corresponding to 4.4 standard diviations \cite{CMS2}. Since the discovery of the process should be possible with the full Run-2 dataset, studies with regard to a differential measurement are carried out in this thesis. 
The differnetial measurement would yield a close examination of the structure of the electromagnetic coupling of the top quark. \\
As $tq\gamma$ is a rare process of the SM, a significant amount of background occupies the measurement region and the signal to background ratio is inherently small. A classifying neural network (NN) is implemented to discriminate the signal process $tq\gamma$ from the background processes. 
This NN is trained on simulated data and receives characteristic event variables of $tq\gamma$ as input. 
Studying the significance and effects of different input parameters on the output may help narrow down or provide conditions for the event selection to optimise background suppression which would be useful for the differential analysis. Additionally, the investigation of these input features could provide vital insights into the nature of the $tq\gamma$ process. 
In this thesis, the correlations of characteristic features of $tq\gamma$ with the NN output are analysed. 
Subsequently, two input features, the transverse momentum of the photon $p_T^\gamma$ and the sum of the forward jet energies and the photon energy, are further studied. 
Changes in the NN output distribution are examined for different requirements on the kinematic variables.

